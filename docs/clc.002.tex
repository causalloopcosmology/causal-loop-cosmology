\documentclass[11pt, a4paper]{article}
\usepackage{amsmath, amssymb, amsthm, geometry, algorithm, algcompatible, graphicx}
\usepackage[utf8]{inputenc}
\usepackage{hyperref}
\geometry{a4paper, margin=1in}

\newtheorem{theorem}{Theorem}[section]
\newtheorem{definition}{Definition}[section]
\newtheorem{proposition}{Proposition}[section]
\newtheorem{lemma}{Lemma}[section]
\newtheorem{remark}{Remark}[section]

\newcommand{\tr}{\mathrm{tr}}
\newcommand{\id}{\mathrm{id}}

\title{\textbf{Thermodynamic Graph Rewriting: A Category-Theoretic Model of Emergent Spacetime}}
\author{}
\date{September 9, 2025}
\maketitle

\begin{abstract}
\noindent We present a mathematically rigorous and computationally explicit model of emergent spacetime, in which a continuous geometric manifold arises from a discrete causal network through a thermodynamically guided graph rewriting process. Spacetime quanta are identified with $3$-cycles, representing the fundamental units of geometric information. We formalize the causal graph $G_t = (V_t, E_t)$ as a category, define the rewrite rule $\mathcal{R}$ as a state-dependent stochastic process, and provide both matrix and algorithmic formulations. A simulation on a 10-node graph demonstrates the nucleation of geometry. We then unify the model’s scaling constants in natural units, showing how local action, curvature, and mass emerge as derived quantities. Finally, we outline a heuristic derivation of the predicted stochastic gravitational-wave spectrum $\Omega_{\mathrm{GW}}(f) \propto f^3$ and argue for the statistical recovery of Lorentz invariance in the continuum limit. This establishes a falsifiable foundation for emergent spacetime in which classical and quantum dynamics are not imposed but derived.
\end{abstract}

\section{The Causal Graph as a Category}

We define the causal graph $G_t = (V_t, E_t)$ at logical time $t$, where $V_t$ is a finite set of events and $E_t \subseteq V_t \times V_t$ a set of directed causal links. The graph is a \textbf{finite directed acyclic graph (DAG)}, ensuring a well-defined causal order and preventing causal paradoxes.

\begin{definition}[Causal Category $\mathcal{C}_t$]
The causal category $\mathcal{C}_t$ is defined by:
\item \textbf{Objects}: $\mathrm{Obj}(\mathcal{C}_t) = V_t$,
    \item \textbf{Morphisms}: finite directed paths $p: v \to w$,
    \item \textbf{Identity}: $\id_v: v \to v$ is the null path (a formal device for category theory, not a physical self-loop),
    \item \textbf{Composition}: path concatenation $q \circ p: u \to w$ for $p: u \to v$, $q: v \to w$.
\end{definition}

\begin{proposition}
$ \mathcal{C}_t $ is a small category.
\end{proposition}
\begin{proof}
Path concatenation is associative, and null paths serve as identities. Since $V_t$ and $E_t$ are finite, both objects and morphisms are finite, so $\mathcal{C}_t$ is small.
\end{proof}

\begin{remark}
The vacuum state is realized as a finite acyclic graph with high symmetry (e.g., a Bethe lattice fragment). The automorphism group $\mathrm{Aut}(G_0)$ acts transitively, ensuring homogeneity of the initial category $\mathcal{C}_0$. This symmetry is preserved under the parallel action of $\mathcal{R}$.
\end{remark}

\begin{proposition}
A 3-cycle $v \to w \to u \to v$ with morphisms $f: v \to w$, $g: w \to u$, and $h: u \to v$ defines a nontrivial endomorphism $\ell_v = h \circ g \circ f$ on $v$.
\end{proposition}
\begin{proof}
$\ell_v$ is a path of length 3 from $v$ to itself, distinct from the null path $\id_v$. It encodes a discrete unit of geometric information.
\end{proof}

\section{The Rewrite Rule as a Stochastic Process}

\begin{definition}[Shortcut Rule]
If $(u \to w)$ and $(w \to v)$ are edges in $E_t$, the rule may add a new edge $(u \to v)$, subject to thermodynamic and geometric constraints.
\end{definition}

\begin{proposition}
The shortcut rule preserves acyclicity.
\end{proposition}
\begin{proof}
Adding $(u,v)$ closes only preexisting two-step paths. Since no path $v \to u$ exists prior to insertion (checked explicitly via BFS), no directed cycle can form. The graph remains a DAG.
\end{proof}

\begin{remark}
The rewrite rule is not a categorical functor in the standard sense, since it is probabilistic and state-dependent. It is better described as a stochastic process acting on categories. This construction parallels causal set dynamics, where the Hasse diagram evolves via random growth.
\end{remark}

\section{Matrix Formulation}

\begin{definition}[Adjacency Matrix]
$A_t \in \{0,1\}^{n \times n}$ with $A_t[i,j] = 1$ iff $(v_i \to v_j) \in E_t$.
\end{definition}

\begin{definition}[3-Cycle Density]
The 3-cycle density at node $v_i$ is
\[
\rho_C(i) = (A_t^3)_{ii} = \sum_{j,k} A_t[i,j] A_t[j,k] A_t[k,i].
\]
The total number of distinct 3-cycles is $N_3 = \frac{1}{3}\tr(A_t^3)$.
\end{definition}

\begin{definition}[Matrix Rewrite Rule]
The update $A_t \mapsto A_{t+1} = A_t + \Delta A_t$ accepts $(u,v)$ if:
\begin{enumerate}
    \item \textbf{Acyclicity}: no path $v \to u$ exists in $G_t$,
    \item \textbf{Geometric Nucleation}: $\Delta \tr(A^3) > 0$ and $\Delta \tr(A^k) = 0$ for $k \ge 4$.
\end{enumerate}
\end{definition}

\begin{example}[5-Node Graph]
For adjacency $A_t$ with $\tr(A_t^3)=0$, adding edge $(5,2)$ closes $2 \to 4 \to 5 \to 2$. Computing powers shows $\tr(A_{t+1}^3)=3$ and no higher cycles. Thus the update is accepted.
\end{example}

\section{Algorithmic Implementation}

\begin{definition}[Causal Boundary and Depth]
The causal boundary $\partial G_t$ consists of nodes with out-degree zero. The causal depth $l_c(v)$ of node $v$ is the length of the shortest path to $\partial G_t$.
\end{definition}

\begin{algorithm}[H]
\caption{Rewrite Rule $\mathcal{R}$: Parallel Update}
\begin{algorithmic}[1]
\Require Graph $G_t=(V_t,E_t)$
\Ensure Updated $G_{t+1}$
\State Pre-compute $\rho_C(v)=(A_t^3)_{vv}$ for all $v$
\State $T \gets 1/\ln(|V_t|)$
\State $\texttt{ProposalList} \gets \emptyset$
\ForAll{$(u,w,v)$ with $u \to w$, $w \to v$}
    \If{$(u,v) \notin E_t$ and $\rho_C(u)+\rho_C(v)>\theta$}
        \State $\Delta \rho_C \gets 1$, $\Delta S \gets 1$
        \State $\Delta F \gets \alpha \Delta \rho_C - T\Delta S$
        \State $P \gets \min(1,\exp(-\Delta F/T))$
        \If{random() $< P$ and is\_permissible($u,v$)}
            \State Add $(u,v)$ to \texttt{ProposalList}
        \EndIf
    \EndIf
\EndFor
\State Return $G_{t+1}$ with all proposed edges added
\end{algorithmic}
\end{algorithm}

\begin{algorithm}[H]
\caption{is\_permissible$(u,v)$}
\begin{algorithmic}[1]
\If{no path $v \to u$ exists}
    \State \Return True
\Else
    \State Form $A_{t+1}$ with $(u,v)$
    \If{$\Delta \tr(A^3)>0$ and $\Delta \tr(A^k)=0$ for $k\ge4$}
        \State \Return True
    \EndIf
\EndIf
\State \Return False
\end{algorithmic}
\end{algorithm}

\section{Simulation on a 10-Node Graph}
We simulate $\mathcal{R}$ on an initial 10-node acyclic graph with average degree 2.8. After 10 update steps, the system undergoes a dynamical transition: from a purely acyclic state to one with four distinct 3-cycles. No cycles of length $>3$ were produced. This demonstrates that maximal parallelism preserves symmetry while permitting nucleation of geometry. Code is available at: \href{https://github.com/causalloopcosmology/causal-loop-cosmology}{\texttt{github.com/causalloopcosmology/causal-loop-cosmology}}.

\section{Unification of Scaling Constants}

In natural units $\hbar=c=l_P=1$, the model’s constants reduce to dimensionless order-unity quantities.

\begin{proposition}[Derived Relations]
\item \textbf{Local Action}: $\mathcal{A}(v) = \rho_C(v) \, l_c(v)$
    \item \textbf{Discrete Curvature}: $R(v) = \sum_{u \in \mathcal{N}(v)} \kappa(v,u)$, with Ollivier–Ricci curvature $\kappa(v,u) = 1 - W_1(m_v,m_u)/d(v,u)$.
    \item \textbf{Mass}: $m(\mathcal{C}) = \sum_{v\in\mathcal{C}} \rho_C(v)$ for a cluster $\mathcal{C}$.
\end{proposition}

\begin{remark}[Falsifiability]
The model makes falsifiable predictions when restored to SI units. The most direct signature is a stochastic GW spectrum $\Omega_{\mathrm{GW}}(f)\propto f^3$ (Sec.~\ref{subsec:gw_spectrum}), testable by detectors across a wide frequency range.
\end{remark}

\subsection{Scaling of the Stochastic Gravitational Wave Background}
\label{subsec:gw_spectrum}
Each rewrite event nucleates a 3-cycle, equivalent to one bit of entropy. The sequence of such events forms a stochastic source of curvature. In Fourier space, uncorrelated events yield a flat spectrum $P(f)\propto f^0$. Curvature insertions act as second derivatives, boosting power by $f^2$. Since GW energy density scales with the square of the strain’s time derivative, an additional factor $f$ arises \cite{Maggiore2007}. Thus,
\[
\Omega_{\mathrm{GW}}(f) \;\propto\; f^3.
\]
This scaling parallels causal set predictions of blue-tilted spectra \cite{Carlip2017}. A full derivation, including coefficients, remains future work.

\subsection{Recovery of Lorentz Invariance}
\label{subsec:lorentz}
Discrete models risk breaking Lorentz invariance. In CST, Poisson sprinkling ensures statistical invariance \cite{Bombelli1987,Dowker2005}. Our rewrite process achieves an analogous effect:
\begin{enumerate}
    \item It is purely local, with no preferred frame.
    \item 3-cycle closure applies isotropically to all paths.
    \item Simulations show narrowing distributions of $\rho_C(i)$, suggesting large-$N$ homogeneity.
\end{enumerate}
We conjecture that in the limit $N \to \infty$, the process converges statistically to a Poisson sprinkling in Minkowski space, ensuring emergent Lorentz invariance. A rigorous proof remains an open priority.

\bibliographystyle{unsrt}
\begin{thebibliography}{9}
\bibitem{Maggiore2007} M. Maggiore, \textit{Gravitational Waves, Volume 1: Theory and Experiments}, Oxford University Press (2007).
\bibitem{Carlip2017} S. Carlip, ``Dimension and Dimensional Reduction in Quantum Gravity,'' \textit{Class. Quant. Grav.} \textbf{34}, 193001 (2017).
\bibitem{Bombelli1987} L. Bombelli, J. Lee, D. Meyer, and R. Sorkin, ``Space-time as a causal set,'' \textit{Phys. Rev. Lett.} \textbf{59}, 521 (1987).
\bibitem{Dowker2005} F. Dowker, ``Causal sets and the deep structure of spacetime,'' in \textit{100 Years of Relativity: Space-Time Structure: Einstein and Beyond}, ed. A. Ashtekar, World Scientific (2005).
\end{thebibliography}

\end{document}