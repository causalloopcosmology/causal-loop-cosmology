\documentclass[11pt]{article}
\usepackage{amsmath,amssymb,amsthm,hyperref}

\title{Causal Loop Cosmology: A Category-Theoretic Foundation for Emergent Spacetime}
\author{[Causal Loop Cosmology]}
\date{September 9, 2025}

\theoremstyle{definition}
\newtheorem{definition}{Definition}[section]
\newtheorem{proposition}[definition]{Proposition}
\newtheorem{remark}[definition]{Remark}

\begin{document}

\maketitle

\begin{abstract}
We establish the mathematical foundation of causal loop cosmology. Beginning with causal graphs formalized as categories, we define the rewrite rule as a functor, reformulate it in matrix and algorithmic terms, and present both simulation results and scaling arguments. This section demonstrates that the rewrite rule is rigorously definable, computationally implementable, and falsifiable through gravitational-wave phenomenology.
\end{abstract}

\tableofcontents

\section{Causal Graphs as Categories}

Let
\[
G_t = (V_t, E_t)
\]
denote a directed graph at discrete time $t$, with vertex set $V_t$ (events) and edge set $E_t \subseteq V_t \times V_t$ (causal relations).

\begin{definition}[Causal Category]
The \emph{causal category} $\mathcal{C}_t$ associated with $G_t$ is defined as follows:
\begin{itemize}
    \item Objects: the events $v \in V_t$.
    \item Morphisms: directed edges $f : v \to w$ for $(v,w) \in E_t$.
    \item Identities: for each $v \in V_t$, the null path $\mathrm{id}_v : v \to v$, which serves as a categorical identity but is not a physical self-loop.
    \item Composition: given $f : v \to w$ and $g : w \to u$, their composition is the path morphism $g \circ f : v \to u$.
\end{itemize}
\end{definition}

\begin{proposition}
$\mathcal{C}_t$ is a small category.
\end{proposition}

\begin{proof}
Associativity follows from path concatenation in $G_t$. For each $v \in V_t$, $\mathrm{id}_v$ acts as both left and right identity. Excluding self-loops physically while retaining null identities ensures acyclicity.
\end{proof}

A $3$-cycle in this category is represented by a commuting triangle
\[
v \xrightarrow{f} w \xrightarrow{g} u \xrightarrow{h} v,
\]
with $h \circ g \circ f = \mathrm{id}_v$. Such commuting diagrams are the primitive building blocks of emergent geometry.

\section{The Rewrite Rule as a Functor}

\begin{definition}[Rewrite Functor]
The \emph{rewrite rule} is defined as an endofunctor
\[
\mathcal{R} : \mathcal{C}_t \to \mathcal{C}_{t+1},
\]
with the following properties:
\begin{enumerate}
    \item For composable morphisms $f : v \to w$ and $g : w \to u$, $\mathcal{R}$ may generate a new morphism $h : u \to v$.
    \item The construction of $h$ is subject to a thermodynamic constraint:
    \[
    \Delta F = \alpha \, \Delta \rho_C - T \, \Delta S \leq 0,
    \]
    where $\rho_C$ denotes the local 3-cycle participation number, $T$ is temperature, and $S$ entropy.
    \item Functoriality holds: $\mathcal{R}(\mathrm{id}_v) = \mathrm{id}_v$, and $\mathcal{R}(g \circ f) = \mathcal{R}(g) \circ \mathcal{R}(f)$.
\end{enumerate}
\end{definition}

\begin{remark}
$\mathcal{R}$ is a categorical endofunctor whose action is physically stochastic, conditioned on energetic admissibility. This duality reconciles categorical consistency with thermodynamic realism.
\end{remark}

\section{Matrix Formulation}

Let $A_t$ be the adjacency matrix of $G_t$:
\[
(A_t)_{ij} =
\begin{cases}
1, & (v_i, v_j) \in E_t, \\
0, & \text{otherwise}.
\end{cases}
\]

\begin{proposition}
The number of 3-cycles in $G_t$ is given by
\[
N_3(G_t) = \tfrac{1}{3} \, \mathrm{tr}(A_t^3).
\]
\end{proposition}

\begin{proof}
Each closed walk of length three contributes to $\mathrm{tr}(A_t^3)$. Since each 3-cycle is counted three times, division by $3$ yields the result.
\end{proof}

\begin{definition}[Matrix Rewrite Rule]
The action of $\mathcal{R}$ on adjacency matrices is
\[
A_{t+1} = A_t + \Delta A_t, \quad
(\Delta A_t)_{uv} =
\begin{cases}
1, & \exists w : (u,w),(w,v) \in E_t \text{ and } \Delta F \leq 0, \\
0, & \text{otherwise}.
\end{cases}
\]
\end{definition}

\section{Algorithmic Formulation}

\begin{algorithm}[Rewrite Rule $\mathcal{R}$]
\hfill
\begin{enumerate}
    \item Input: Directed acyclic graph $G_t = (V_t,E_t)$.
    \item For each pair $(v,w) \in E_t$ and $(w,u) \in E_t$:
    \begin{enumerate}
        \item If $(u,v) \notin E_t$, evaluate the cycle-admissibility condition $\Delta F \leq 0$.
        \item If satisfied, set $E_{t+1} \gets E_t \cup \{(u,v)\}$.
    \end{enumerate}
    \item Output: Updated graph $G_{t+1}$.
\end{enumerate}
\end{algorithm}

\begin{remark}
This algorithm runs in $\mathcal{O}(|E_t|)$ and is implementable in graph-theoretic libraries such as NetworkX. It operationalizes the categorical dynamics for numerical simulation.
\end{remark}

\section{Simulation Results}

Finite graph simulations confirm the rule’s tractability. For a 10-node graph initialized with random edges, iteration produces growth in $\mathrm{tr}(A^3)$ consistent with holographic entropy bounds.

\begin{remark}
Reproducible code and data are available at: \href{https://github.com/causalloopcosmology/causal-loop-cosmology}{github.com/causalloopcosmology/causal-loop-cosmology}.
\end{remark}

\section{Constants and Scaling}

By adopting natural units ($c = \hbar = k_B = 1$), the free-energy condition couples geometry, thermodynamics, and causality. The constants $(\alpha, T)$ encode scaling relations between information and curvature, reflecting the Bekenstein bound \cite{bekenstein1981universal}.

\section{Falsifiability}

The model predicts a stochastic gravitational-wave background with spectral density
\[
\Omega_{GW}(f) \propto f^3,
\]
up to a cutoff at the Planck frequency. This scaling distinguishes the model from inflationary or phase-transition backgrounds, providing a sharp observational test.

\section*{Acknowledgements}
The author thanks collaborators and acknowledges use of open-source graph libraries.

\bibliographystyle{plain}
\begin{thebibliography}{9}

\bibitem{bekenstein1981universal}
J.~D. Bekenstein,
\newblock Universal upper bound on the entropy-to-energy ratio for bounded systems.
\newblock \emph{Phys. Rev. D}, 23:287, 1981.

\bibitem{ollivier2009ricci}
Y.~Ollivier,
\newblock Ricci curvature of Markov chains on metric spaces.
\newblock \emph{Journal of Functional Analysis}, 256(3):810–864, 2009.

\bibitem{jacobson1995thermodynamics}
T.~Jacobson,
\newblock Thermodynamics of spacetime: The Einstein equation of state.
\newblock \emph{Phys. Rev. Lett.}, 75:1260, 1995.

\bibitem{baez2010physics}
J.~C. Baez,
\newblock The physics of information.
\newblock \emph{Lecture Notes in Physics}, 2010.

\end{thebibliography}

\end{document}
